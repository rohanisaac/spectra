% Generated by Sphinx.
\def\sphinxdocclass{report}
\documentclass[letterpaper,10pt,english]{sphinxmanual}
\usepackage[utf8]{inputenc}
\DeclareUnicodeCharacter{00A0}{\nobreakspace}
\usepackage{cmap}
\usepackage[T1]{fontenc}
\usepackage{babel}
\usepackage{times}
\usepackage[Bjarne]{fncychap}
\usepackage{longtable}
\usepackage{sphinx}
\usepackage{multirow}


\title{Spectra Documentation}
\date{June 15, 2014}
\release{0.5}
\author{Rohan Isaac}
\newcommand{\sphinxlogo}{}
\renewcommand{\releasename}{Release}
\makeindex

\makeatletter
\def\PYG@reset{\let\PYG@it=\relax \let\PYG@bf=\relax%
    \let\PYG@ul=\relax \let\PYG@tc=\relax%
    \let\PYG@bc=\relax \let\PYG@ff=\relax}
\def\PYG@tok#1{\csname PYG@tok@#1\endcsname}
\def\PYG@toks#1+{\ifx\relax#1\empty\else%
    \PYG@tok{#1}\expandafter\PYG@toks\fi}
\def\PYG@do#1{\PYG@bc{\PYG@tc{\PYG@ul{%
    \PYG@it{\PYG@bf{\PYG@ff{#1}}}}}}}
\def\PYG#1#2{\PYG@reset\PYG@toks#1+\relax+\PYG@do{#2}}

\expandafter\def\csname PYG@tok@gd\endcsname{\def\PYG@tc##1{\textcolor[rgb]{0.63,0.00,0.00}{##1}}}
\expandafter\def\csname PYG@tok@gu\endcsname{\let\PYG@bf=\textbf\def\PYG@tc##1{\textcolor[rgb]{0.50,0.00,0.50}{##1}}}
\expandafter\def\csname PYG@tok@gt\endcsname{\def\PYG@tc##1{\textcolor[rgb]{0.00,0.27,0.87}{##1}}}
\expandafter\def\csname PYG@tok@gs\endcsname{\let\PYG@bf=\textbf}
\expandafter\def\csname PYG@tok@gr\endcsname{\def\PYG@tc##1{\textcolor[rgb]{1.00,0.00,0.00}{##1}}}
\expandafter\def\csname PYG@tok@cm\endcsname{\let\PYG@it=\textit\def\PYG@tc##1{\textcolor[rgb]{0.25,0.50,0.56}{##1}}}
\expandafter\def\csname PYG@tok@vg\endcsname{\def\PYG@tc##1{\textcolor[rgb]{0.73,0.38,0.84}{##1}}}
\expandafter\def\csname PYG@tok@m\endcsname{\def\PYG@tc##1{\textcolor[rgb]{0.13,0.50,0.31}{##1}}}
\expandafter\def\csname PYG@tok@mh\endcsname{\def\PYG@tc##1{\textcolor[rgb]{0.13,0.50,0.31}{##1}}}
\expandafter\def\csname PYG@tok@cs\endcsname{\def\PYG@tc##1{\textcolor[rgb]{0.25,0.50,0.56}{##1}}\def\PYG@bc##1{\setlength{\fboxsep}{0pt}\colorbox[rgb]{1.00,0.94,0.94}{\strut ##1}}}
\expandafter\def\csname PYG@tok@ge\endcsname{\let\PYG@it=\textit}
\expandafter\def\csname PYG@tok@vc\endcsname{\def\PYG@tc##1{\textcolor[rgb]{0.73,0.38,0.84}{##1}}}
\expandafter\def\csname PYG@tok@il\endcsname{\def\PYG@tc##1{\textcolor[rgb]{0.13,0.50,0.31}{##1}}}
\expandafter\def\csname PYG@tok@go\endcsname{\def\PYG@tc##1{\textcolor[rgb]{0.20,0.20,0.20}{##1}}}
\expandafter\def\csname PYG@tok@cp\endcsname{\def\PYG@tc##1{\textcolor[rgb]{0.00,0.44,0.13}{##1}}}
\expandafter\def\csname PYG@tok@gi\endcsname{\def\PYG@tc##1{\textcolor[rgb]{0.00,0.63,0.00}{##1}}}
\expandafter\def\csname PYG@tok@gh\endcsname{\let\PYG@bf=\textbf\def\PYG@tc##1{\textcolor[rgb]{0.00,0.00,0.50}{##1}}}
\expandafter\def\csname PYG@tok@ni\endcsname{\let\PYG@bf=\textbf\def\PYG@tc##1{\textcolor[rgb]{0.84,0.33,0.22}{##1}}}
\expandafter\def\csname PYG@tok@nl\endcsname{\let\PYG@bf=\textbf\def\PYG@tc##1{\textcolor[rgb]{0.00,0.13,0.44}{##1}}}
\expandafter\def\csname PYG@tok@nn\endcsname{\let\PYG@bf=\textbf\def\PYG@tc##1{\textcolor[rgb]{0.05,0.52,0.71}{##1}}}
\expandafter\def\csname PYG@tok@no\endcsname{\def\PYG@tc##1{\textcolor[rgb]{0.38,0.68,0.84}{##1}}}
\expandafter\def\csname PYG@tok@na\endcsname{\def\PYG@tc##1{\textcolor[rgb]{0.25,0.44,0.63}{##1}}}
\expandafter\def\csname PYG@tok@nb\endcsname{\def\PYG@tc##1{\textcolor[rgb]{0.00,0.44,0.13}{##1}}}
\expandafter\def\csname PYG@tok@nc\endcsname{\let\PYG@bf=\textbf\def\PYG@tc##1{\textcolor[rgb]{0.05,0.52,0.71}{##1}}}
\expandafter\def\csname PYG@tok@nd\endcsname{\let\PYG@bf=\textbf\def\PYG@tc##1{\textcolor[rgb]{0.33,0.33,0.33}{##1}}}
\expandafter\def\csname PYG@tok@ne\endcsname{\def\PYG@tc##1{\textcolor[rgb]{0.00,0.44,0.13}{##1}}}
\expandafter\def\csname PYG@tok@nf\endcsname{\def\PYG@tc##1{\textcolor[rgb]{0.02,0.16,0.49}{##1}}}
\expandafter\def\csname PYG@tok@si\endcsname{\let\PYG@it=\textit\def\PYG@tc##1{\textcolor[rgb]{0.44,0.63,0.82}{##1}}}
\expandafter\def\csname PYG@tok@s2\endcsname{\def\PYG@tc##1{\textcolor[rgb]{0.25,0.44,0.63}{##1}}}
\expandafter\def\csname PYG@tok@vi\endcsname{\def\PYG@tc##1{\textcolor[rgb]{0.73,0.38,0.84}{##1}}}
\expandafter\def\csname PYG@tok@nt\endcsname{\let\PYG@bf=\textbf\def\PYG@tc##1{\textcolor[rgb]{0.02,0.16,0.45}{##1}}}
\expandafter\def\csname PYG@tok@nv\endcsname{\def\PYG@tc##1{\textcolor[rgb]{0.73,0.38,0.84}{##1}}}
\expandafter\def\csname PYG@tok@s1\endcsname{\def\PYG@tc##1{\textcolor[rgb]{0.25,0.44,0.63}{##1}}}
\expandafter\def\csname PYG@tok@gp\endcsname{\let\PYG@bf=\textbf\def\PYG@tc##1{\textcolor[rgb]{0.78,0.36,0.04}{##1}}}
\expandafter\def\csname PYG@tok@sh\endcsname{\def\PYG@tc##1{\textcolor[rgb]{0.25,0.44,0.63}{##1}}}
\expandafter\def\csname PYG@tok@ow\endcsname{\let\PYG@bf=\textbf\def\PYG@tc##1{\textcolor[rgb]{0.00,0.44,0.13}{##1}}}
\expandafter\def\csname PYG@tok@sx\endcsname{\def\PYG@tc##1{\textcolor[rgb]{0.78,0.36,0.04}{##1}}}
\expandafter\def\csname PYG@tok@bp\endcsname{\def\PYG@tc##1{\textcolor[rgb]{0.00,0.44,0.13}{##1}}}
\expandafter\def\csname PYG@tok@c1\endcsname{\let\PYG@it=\textit\def\PYG@tc##1{\textcolor[rgb]{0.25,0.50,0.56}{##1}}}
\expandafter\def\csname PYG@tok@kc\endcsname{\let\PYG@bf=\textbf\def\PYG@tc##1{\textcolor[rgb]{0.00,0.44,0.13}{##1}}}
\expandafter\def\csname PYG@tok@c\endcsname{\let\PYG@it=\textit\def\PYG@tc##1{\textcolor[rgb]{0.25,0.50,0.56}{##1}}}
\expandafter\def\csname PYG@tok@mf\endcsname{\def\PYG@tc##1{\textcolor[rgb]{0.13,0.50,0.31}{##1}}}
\expandafter\def\csname PYG@tok@err\endcsname{\def\PYG@bc##1{\setlength{\fboxsep}{0pt}\fcolorbox[rgb]{1.00,0.00,0.00}{1,1,1}{\strut ##1}}}
\expandafter\def\csname PYG@tok@kd\endcsname{\let\PYG@bf=\textbf\def\PYG@tc##1{\textcolor[rgb]{0.00,0.44,0.13}{##1}}}
\expandafter\def\csname PYG@tok@ss\endcsname{\def\PYG@tc##1{\textcolor[rgb]{0.32,0.47,0.09}{##1}}}
\expandafter\def\csname PYG@tok@sr\endcsname{\def\PYG@tc##1{\textcolor[rgb]{0.14,0.33,0.53}{##1}}}
\expandafter\def\csname PYG@tok@mo\endcsname{\def\PYG@tc##1{\textcolor[rgb]{0.13,0.50,0.31}{##1}}}
\expandafter\def\csname PYG@tok@mi\endcsname{\def\PYG@tc##1{\textcolor[rgb]{0.13,0.50,0.31}{##1}}}
\expandafter\def\csname PYG@tok@kn\endcsname{\let\PYG@bf=\textbf\def\PYG@tc##1{\textcolor[rgb]{0.00,0.44,0.13}{##1}}}
\expandafter\def\csname PYG@tok@o\endcsname{\def\PYG@tc##1{\textcolor[rgb]{0.40,0.40,0.40}{##1}}}
\expandafter\def\csname PYG@tok@kr\endcsname{\let\PYG@bf=\textbf\def\PYG@tc##1{\textcolor[rgb]{0.00,0.44,0.13}{##1}}}
\expandafter\def\csname PYG@tok@s\endcsname{\def\PYG@tc##1{\textcolor[rgb]{0.25,0.44,0.63}{##1}}}
\expandafter\def\csname PYG@tok@kp\endcsname{\def\PYG@tc##1{\textcolor[rgb]{0.00,0.44,0.13}{##1}}}
\expandafter\def\csname PYG@tok@w\endcsname{\def\PYG@tc##1{\textcolor[rgb]{0.73,0.73,0.73}{##1}}}
\expandafter\def\csname PYG@tok@kt\endcsname{\def\PYG@tc##1{\textcolor[rgb]{0.56,0.13,0.00}{##1}}}
\expandafter\def\csname PYG@tok@sc\endcsname{\def\PYG@tc##1{\textcolor[rgb]{0.25,0.44,0.63}{##1}}}
\expandafter\def\csname PYG@tok@sb\endcsname{\def\PYG@tc##1{\textcolor[rgb]{0.25,0.44,0.63}{##1}}}
\expandafter\def\csname PYG@tok@k\endcsname{\let\PYG@bf=\textbf\def\PYG@tc##1{\textcolor[rgb]{0.00,0.44,0.13}{##1}}}
\expandafter\def\csname PYG@tok@se\endcsname{\let\PYG@bf=\textbf\def\PYG@tc##1{\textcolor[rgb]{0.25,0.44,0.63}{##1}}}
\expandafter\def\csname PYG@tok@sd\endcsname{\let\PYG@it=\textit\def\PYG@tc##1{\textcolor[rgb]{0.25,0.44,0.63}{##1}}}

\def\PYGZbs{\char`\\}
\def\PYGZus{\char`\_}
\def\PYGZob{\char`\{}
\def\PYGZcb{\char`\}}
\def\PYGZca{\char`\^}
\def\PYGZam{\char`\&}
\def\PYGZlt{\char`\<}
\def\PYGZgt{\char`\>}
\def\PYGZsh{\char`\#}
\def\PYGZpc{\char`\%}
\def\PYGZdl{\char`\$}
\def\PYGZhy{\char`\-}
\def\PYGZsq{\char`\'}
\def\PYGZdq{\char`\"}
\def\PYGZti{\char`\~}
% for compatibility with earlier versions
\def\PYGZat{@}
\def\PYGZlb{[}
\def\PYGZrb{]}
\makeatother

\begin{document}

\maketitle
\tableofcontents
\phantomsection\label{index::doc}


Contents:


\chapter{Spectra class}
\label{spectra:spectra-class}\label{spectra::doc}\label{spectra:welcome-to-spectra-s-documentation}\index{Spectra (class in spectra)}

\begin{fulllineitems}
\phantomsection\label{spectra:spectra.Spectra}\pysiglinewithargsret{\strong{class }\code{spectra.}\bfcode{Spectra}}{\emph{filename}}{}
Stores spectra data in Spec (peak-o-mat) format
\paragraph{Methods}
\index{build\_model() (spectra.Spectra method)}

\begin{fulllineitems}
\phantomsection\label{spectra:spectra.Spectra.build_model}\pysiglinewithargsret{\bfcode{build\_model}}{\emph{peak\_type='LO'}, \emph{max\_width=None}}{}
Builds a peak-o-mat model of peaks in listed by index in \emph{peak\_pos}
\begin{quote}\begin{description}
\item[{Parameters}] \leavevmode
\textbf{peak\_type} : string
\begin{quote}

Peaks can be of the following types: 
(to setup custom peaks and more, see peak-o-mat docs)

\begin{DUlineblock}{0em}
\item[] `LO' : symmetric lorentzian
\item[] `GA' : symmetric gaussain
\item[] `VO' : voigt profile
\item[] `PVO' : psuedo-voigt profile
\item[] `FAN' : fano lineshape        
\end{DUlineblock}
\end{quote}

\textbf{Peaks can veof default type lorentzian (LO). Uses some basic algorithms} :

\textbf{to determine initial parameters for amplitude and fwhm (limit on fwhm} :

\textbf{to avoid fitting background as peaks.} :

\end{description}\end{quote}

\end{fulllineitems}

\index{filter\_high\_freq() (spectra.Spectra method)}

\begin{fulllineitems}
\phantomsection\label{spectra:spectra.Spectra.filter_high_freq}\pysiglinewithargsret{\bfcode{filter\_high\_freq}}{\emph{data}}{}
Filter high frequency data using fft
\begin{quote}\begin{description}
\item[{Parameters}] \leavevmode
\textbf{data} : np.array
\begin{quote}

Data to filter
\end{quote}

\item[{Returns}] \leavevmode
\textbf{data} : np.array
\begin{quote}

Data with high frequcny components removed
\end{quote}

\end{description}\end{quote}

\end{fulllineitems}

\index{find\_background() (spectra.Spectra method)}

\begin{fulllineitems}
\phantomsection\label{spectra:spectra.Spectra.find_background}\pysiglinewithargsret{\bfcode{find\_background}}{\emph{sub\_range=None}, \emph{poly\_deg=3}, \emph{smoothing=5}}{}
Attempts to find the background of the spectra, 
and updates the \emph{bg} array

\end{fulllineitems}

\index{find\_fwhm() (spectra.Spectra method)}

\begin{fulllineitems}
\phantomsection\label{spectra:spectra.Spectra.find_fwhm}\pysiglinewithargsret{\bfcode{find\_fwhm}}{\emph{position}}{}
Find the fwhm of a point using a very simplisitic algorigthm. 
Could return very large width.

\end{fulllineitems}

\index{find\_peaks() (spectra.Spectra method)}

\begin{fulllineitems}
\phantomsection\label{spectra:spectra.Spectra.find_peaks}\pysiglinewithargsret{\bfcode{find\_peaks}}{\emph{lower=None}, \emph{upper=None}, \emph{threshold=5}, \emph{limit=20}}{}
Find peaks in actve data set using continous wavelet 
transformation from \emph{scipy.signal}

\end{fulllineitems}

\index{fit\_data() (spectra.Spectra method)}

\begin{fulllineitems}
\phantomsection\label{spectra:spectra.Spectra.fit_data}\pysiglinewithargsret{\bfcode{fit\_data}}{}{}
Attempt to fit data using peak-o-mat Fit function with the 
generated model. Updates model with fit parameters.

\end{fulllineitems}

\index{guess\_peak\_width() (spectra.Spectra method)}

\begin{fulllineitems}
\phantomsection\label{spectra:spectra.Spectra.guess_peak_width}\pysiglinewithargsret{\bfcode{guess\_peak\_width}}{\emph{max\_width=50}}{}
Find an initial guess for the peak with of the data imported, 
use in peak finding and model buildings and other major functions, 
probably should call in the constructor
\begin{quote}\begin{description}
\item[{Parameters}] \leavevmode
\textbf{max\_width} : int
\begin{quote}

Max width of peaks to search for
\end{quote}

\end{description}\end{quote}
\paragraph{Notes}

Locates the max value in the data
Finds the peak width associated with this data

\end{fulllineitems}

\index{output\_results() (spectra.Spectra method)}

\begin{fulllineitems}
\phantomsection\label{spectra:spectra.Spectra.output_results}\pysiglinewithargsret{\bfcode{output\_results}}{}{}
Output fit paramters as csv values with errors

\end{fulllineitems}

\index{remove\_spikes() (spectra.Spectra method)}

\begin{fulllineitems}
\phantomsection\label{spectra:spectra.Spectra.remove_spikes}\pysiglinewithargsret{\bfcode{remove\_spikes}}{\emph{strength=0.5}}{}
Attempts to remove spikes in active set using a simple test of 
the pixels around it. Fractional value of strength needed.

\end{fulllineitems}

\index{subtract\_background() (spectra.Spectra method)}

\begin{fulllineitems}
\phantomsection\label{spectra:spectra.Spectra.subtract_background}\pysiglinewithargsret{\bfcode{subtract\_background}}{}{}
Subtract background from active spectra

\end{fulllineitems}


\end{fulllineitems}



\chapter{Savitzky Golay}
\label{savitzky_golay:savitzky-golay}\label{savitzky_golay::doc}\index{savitzky\_golay() (in module savitzky\_golay)}

\begin{fulllineitems}
\phantomsection\label{savitzky_golay:savitzky_golay.savitzky_golay}\pysiglinewithargsret{\code{savitzky\_golay.}\bfcode{savitzky\_golay}}{\emph{y}, \emph{window\_size}, \emph{order}, \emph{deriv=0}, \emph{rate=1}}{}
Smooth (and optionally differentiate) data with a Savitzky-Golay filter.
The Savitzky-Golay filter removes high frequency noise from data.
It has the advantage of preserving the original shape and
features of the signal better than other types of filtering
approaches, such as moving averages techniques.
\begin{quote}\begin{description}
\item[{Parameters}] \leavevmode
\textbf{y} : array\_like, shape (N,)
\begin{quote}

the values of the time history of the signal.
\end{quote}

\textbf{window\_size} : int
\begin{quote}

the length of the window. Must be an odd integer number.
\end{quote}

\textbf{order} : int
\begin{quote}

the order of the polynomial used in the filtering.
Must be less then \emph{window\_size} - 1.
\end{quote}

\textbf{deriv: int} :
\begin{quote}

the order of the derivative to compute (default = 0 means only smoothing)
\end{quote}

\item[{Returns}] \leavevmode
\textbf{ys} : ndarray, shape (N)
\begin{quote}

the smoothed signal (or it's n-th derivative).
\end{quote}

\end{description}\end{quote}
\paragraph{Notes}

The Savitzky-Golay is a type of low-pass filter, particularly
suited for smoothing noisy data. The main idea behind this
approach is to make for each point a least-square fit with a
polynomial of high order over a odd-sized window centered at
the point.
\paragraph{References}

{\hyperref[savitzky_golay:r1]{{[}R1{]}}}, {\hyperref[savitzky_golay:r2]{{[}R2{]}}}
\paragraph{Examples}

\begin{Verbatim}[commandchars=\\\{\}]
\PYG{n}{t} \PYG{o}{=} \PYG{n}{np}\PYG{o}{.}\PYG{n}{linspace}\PYG{p}{(}\PYG{o}{\PYGZhy{}}\PYG{l+m+mi}{4}\PYG{p}{,} \PYG{l+m+mi}{4}\PYG{p}{,} \PYG{l+m+mi}{500}\PYG{p}{)}
\PYG{n}{y} \PYG{o}{=} \PYG{n}{np}\PYG{o}{.}\PYG{n}{exp}\PYG{p}{(} \PYG{o}{\PYGZhy{}}\PYG{n}{t}\PYG{o}{*}\PYG{o}{*}\PYG{l+m+mi}{2} \PYG{p}{)} \PYG{o}{+} \PYG{n}{np}\PYG{o}{.}\PYG{n}{random}\PYG{o}{.}\PYG{n}{normal}\PYG{p}{(}\PYG{l+m+mi}{0}\PYG{p}{,} \PYG{l+m+mf}{0.05}\PYG{p}{,} \PYG{n}{t}\PYG{o}{.}\PYG{n}{shape}\PYG{p}{)}
\PYG{n}{ysg} \PYG{o}{=} \PYG{n}{savitzky\PYGZus{}golay}\PYG{p}{(}\PYG{n}{y}\PYG{p}{,} \PYG{n}{window\PYGZus{}size}\PYG{o}{=}\PYG{l+m+mi}{31}\PYG{p}{,} \PYG{n}{order}\PYG{o}{=}\PYG{l+m+mi}{4}\PYG{p}{)}
\PYG{k+kn}{import} \PYG{n+nn}{matplotlib.pyplot} \PYG{k+kn}{as} \PYG{n+nn}{plt}
\PYG{n}{plt}\PYG{o}{.}\PYG{n}{plot}\PYG{p}{(}\PYG{n}{t}\PYG{p}{,} \PYG{n}{y}\PYG{p}{,} \PYG{n}{label}\PYG{o}{=}\PYG{l+s}{\PYGZsq{}}\PYG{l+s}{Noisy signal}\PYG{l+s}{\PYGZsq{}}\PYG{p}{)}
\PYG{n}{plt}\PYG{o}{.}\PYG{n}{plot}\PYG{p}{(}\PYG{n}{t}\PYG{p}{,} \PYG{n}{np}\PYG{o}{.}\PYG{n}{exp}\PYG{p}{(}\PYG{o}{\PYGZhy{}}\PYG{n}{t}\PYG{o}{*}\PYG{o}{*}\PYG{l+m+mi}{2}\PYG{p}{)}\PYG{p}{,} \PYG{l+s}{\PYGZsq{}}\PYG{l+s}{k}\PYG{l+s}{\PYGZsq{}}\PYG{p}{,} \PYG{n}{lw}\PYG{o}{=}\PYG{l+m+mf}{1.5}\PYG{p}{,} \PYG{n}{label}\PYG{o}{=}\PYG{l+s}{\PYGZsq{}}\PYG{l+s}{Original signal}\PYG{l+s}{\PYGZsq{}}\PYG{p}{)}
\PYG{n}{plt}\PYG{o}{.}\PYG{n}{plot}\PYG{p}{(}\PYG{n}{t}\PYG{p}{,} \PYG{n}{ysg}\PYG{p}{,} \PYG{l+s}{\PYGZsq{}}\PYG{l+s}{r}\PYG{l+s}{\PYGZsq{}}\PYG{p}{,} \PYG{n}{label}\PYG{o}{=}\PYG{l+s}{\PYGZsq{}}\PYG{l+s}{Filtered signal}\PYG{l+s}{\PYGZsq{}}\PYG{p}{)}
\PYG{n}{plt}\PYG{o}{.}\PYG{n}{legend}\PYG{p}{(}\PYG{p}{)}
\PYG{n}{plt}\PYG{o}{.}\PYG{n}{show}\PYG{p}{(}\PYG{p}{)}
\end{Verbatim}

\end{fulllineitems}



\chapter{Driver function}
\label{driver_function:driver-function}\label{driver_function::doc}
Example of how to use the Spectra


\section{Import a file}
\label{driver_function:import-a-file}

\section{Find Peaks}
\label{driver_function:find-peaks}
Me


\chapter{Indices and tables}
\label{index:indices-and-tables}\begin{itemize}
\item {} 
\emph{genindex}

\item {} 
\emph{search}

\end{itemize}

\begin{thebibliography}{R1}
\bibitem[R1]{R1}{\phantomsection\label{savitzky_golay:r1} 
A. Savitzky, M. J. E. Golay, Smoothing and Differentiation of
Data by Simplified Least Squares Procedures. Analytical
Chemistry, 1964, 36 (8), pp 1627-1639.
}
\bibitem[R2]{R2}{\phantomsection\label{savitzky_golay:r2} 
Numerical Recipes 3rd Edition: The Art of Scientific Computing
W.H. Press, S.A. Teukolsky, W.T. Vetterling, B.P. Flannery
Cambridge University Press ISBN-13: 9780521880688
}
\end{thebibliography}



\renewcommand{\indexname}{Index}
\printindex
\end{document}
